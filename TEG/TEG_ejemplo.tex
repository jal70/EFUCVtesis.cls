%%%%%%%%%%%%%%%%%%%%%%%%%%%%%%%%%%%%%%%%%%%%%%%%%%%%%%%%%%%%%%%%%%%%%%%%%%%%%%%%%%%%%%%%%%%%%%%
%%%%%%%%%%%%%%%%  Documento de ejemplo para aplicar formato de TEG (Escuela de Físca de la UCV)
%%%%%%%%%%%%%%%%%%%%%%%%%%%%%%%%%%%%%%%%%%%%%%%%%%%%%%%%%%%%%%%%%%%%%%%%%%%%%%%%%%%%%%%%%%%%%%%
%
% Este ejemplo ha sido elaborado por Jose Antonio López. Dudas: jal.ccs@gmail.com
% Está basado en el cls EFUCVtesis.cls
% EFUCVtesis.cls ajusta los márgenes del documento, define babel spanish y fuentes utf8, que permiten usar 
% acentos. También define funciones para manejar la información del autor, título, tutores e 
% incluye las portadas. Los comentarios dan instrucciones sobre el uso de dichas funciones
%%
%%%%%%%%%%%%%%%%%%%%%%%%%%%%%%%%%%%%%%%%%%%%%%%%%%%%%%%%%%%%%%%%%%%%%%%%%%%%%%%%%%%%%%%%%%%%%
%%%%%%%%%%%%%%															%%%%%%%%%%%%%%%%%%%%
%%%%%%%%%%%%%% 						Use pdflatex!						%%%%%%%%%%%%%%%%%%%%
%%%%%%%%%%%%%%															%%%%%%%%%%%%%%%%%%%%
%%%%%%%%%%%%		para generar la biblografía: pdflatex + bibtex + pdflatex		%%%%%%%%%%%%%%%%
%%%%%%%%%%%%%%															%%%%%%%%%%%%%%%%%%%%
%%%%%%%%%%  escrito y probado en texmaker 3.2 y 3.4, bajo ubuntu 12.04		%%%%%%%%%%%%%%%%
%%%%%%%%%%%%%%															%%%%%%%%%%%%%%%%%%%%
%%%%%%%%%%%%%%	probado en el terminal con $pdflatex mi_tesis.tex 		%%%%%%%%%%%%%%%%%%%%
%%%%%%%%%%%%%%							  $bibtex	mi_tesis.aux			%%%%%%%%%%%%%%%%%%%%
%%%%%%%%%%%%%%	 está libre de errores en las condiciones descritas  	%%%%%%%%%%%%%%%%%%%%
%%%%%%%%%%%%%%							 								 %%%%%%%%%%%%%%%%%%%%
%%%%%%%%%%%%%% recomedación: actualizar TexLive antes de usar EFUCVtesis %%%%%%%%%%%%%%%%%%%%
%%%%%%%%%%%%%%%%%%%%%%%%%%%%%%%%%%%%%%%%%%%%%%%%%%%%%%%%%%%%%%%%%%%%%%%%%%%%%%%%%%%%%%%%%%%%%
%%%%%%%%%%%%%%%%%%%%%%%%%%%%%%%%%%%%%%%%%%%%%%%%%%%%%%%%%%%%%%%%%%%%%%%%%%%%%%%%%%%%%%%%%%%%%
%
%%%%%%%%%%%%%%%%%%%%%%%%%%%%%%%%%%%%%%%%%%%%%%%%%%%%%%%%%%%%%%
%%%%%%%%%%%%%%%%%%%%%%%%%%%%%%%%%%%%%%%%%%%%%%%%%%%%%%%%%%%%%%
\documentclass{EFUCVtesis} % importa el cls con los estilos para los TEG de la EF de la UCV
%
%
\titulo{Estudio del Superconductor Intermetálico ternario \NoCaseChange{\ce{La3Pd4Si4}} mediante la dependencia con la temperatura de la longitud de penetración magnética en dimensión \NoCaseChange{$\mathbf{d=3+1}$}}
%%%%%%%%%%%%%%%
% El comando \NoCaseChange{} protege al texto interno de la modificación a mayúsculas. Usar cuando el cambio a mayúsculas pueda afectar la escritura correcta. Ejemplo, compuestos químicos, expresiones matemáticas, etc.
%%%%%%%%%%%%%%%
%
\autor{Nomen Nescio}{m}{1234567} %nombre del autor, género m o f, número de cédula
\anho{Mayo-2015} %mes y año de la presentación (relevante solo en la versión final)
%
%Tutores
%se coloca nombre, género (m o f), título académico, afiliación y cédula
%tutorA aparece primero que TutorB en la portada. Cuando actúen dos tutores la designación A o B es a gusto del consumidor. Se puede usar orden alfabético o cualquier otro orden. 
\tutorA{Fulano de Tal}{m}{Prof.}{UCV}{3141593} 
\tutorB{Sutanita}{f}{M.Sc.}{USB}{2718281} 
\dostutores{no}% si hay dos tutores se coloca si 
%
%
\begin{document}

\pagestyle{empty}
\portada %fabrica la portada usar, solo se imprime en versión final
%
\cleardoublepage
\primerapagina %fabrica la primera página (es igual a la portada, sin el marco)
%
\cleardoublepage
\veredicto{borrador} %Si se escoge \veredicto{final}, se renumeran la páginas para incluir el veredicto luego de la portada. Si se escoge \veredicto{borrador} se inserta la carta de autorización del tutor. Para cualquier otra elección inserta una página con textoprueba.tex
%
%\input{textoveredicto.tex}
\cleardoublepage
\section*{Página libre}
\begin{flushleft}
Las rosas son rojas\\
las violetas, azules...
\end{flushleft}
 % Página para incluir cualquier idea o cita que el estudiante desee y no tenga lugar en el resto del documento
%
\cleardoublepage
\section*{Agradecimientos}
A mí, a ti, a ellos
 % Agradecimientos
%
\cleardoublepage


\begin{letter}

\begin{center}
{\bfseries \uppercase{\expandafter{Resumen}}}
\end{center}


Lorem ipsum dolor sit amet, consectetur adipiscing elit. Quisque at tempor justo. Nullam dolor sapien, rutrum a ultrices eget, malesuada in est. Etiam et ipsum imperdiet, sollicitudin erat at, vehicula dui. Donec faucibus adipiscing magna vel ultrices. Morbi at urna et velit hendrerit consectetur vel eget erat. Morbi auctor imperdiet odio, in consequat neque varius ut. Pellentesque habitant morbi tristique senectus et netus et malesuada fames ac turpis egestas. Mauris mattis consectetur quam, vel fringilla lorem ullamcorper eu. Sed et arcu ante. Nulla pretium tellus id elit egestas, in pulvinar eros ornare. Suspendisse facilisis urna feugiat diam fringilla, eu viverra est viverra. Aliquam diam metus, mattis sit amet lorem id, blandit sagittis magna. Cras eu adipiscing lacus. Mauris ullamcorper tempor eros ac rhoncus. Ut scelerisque magna nibh, sed faucibus libero scelerisque eu. Aliquam ornare lacus eu metus sagittis, non condimentum mauris elementum.

%%%%%%%%%%%%%%%%%%%%%%%%%%%%%%%
%% Quitar este bloque en versión final
%%%%%%%%%%%%%%%%%%%%%%%%%%%%%%%
\begin{center}
{\bf \Large Hasta 400 palabras}
\end{center}
%%%%%%%%%%%%%%%%%%%%%%%%%%%%%
%%
%%%%%%%%%%%%%%%%%%%%%%%%%%%%%

\end{letter}
 %resumen menor a 200 palabras
\cleardoublepage

\tableofcontents % indice de contenidos

\cleardoublepage
\addcontentsline{toc}{chapter}{Lista de figuras} % para que aparezca en el indice de contenidos
\listoffigures % indice de figuras

\cleardoublepage
\addcontentsline{toc}{chapter}{Lista de tablas} % para que aparezca en el indice de contenidos
\listoftables % indice de tablas
% este bloque cambia el estilo de numeracion de paginas de empty al fancy configurado en EFUCVtesis.cls
\newpage
\pagestyle{fancy}
\thispagestyle{empty}
%fin del cambio de estilo de numeración de páginas

%%%%%%%%%%%%%%%%%%
%% aquí se comienza a incluir los capítulos
%%%%%%%%%%%%%%%%
%
\chapter{Lorem ipsum dolor}

Lorem ipsum dolor sit amet, consectetur adipiscing elit \cite{Ade:2014xna}. Quisque at tempor justo. Nullam dolor sapien, rutrum a ultrices eget, malesuada in est. Etiam et ipsum imperdiet, sollicitudin erat at, vehicula dui. Donec faucibus adipiscing magna vel ultrices. Morbi at urna et velit hendrerit consectetur vel eget erat. Morbi auctor imperdiet odio, in consequat neque varius ut. Pellentesque habitant morbi tristique senectus et netus et malesuada fames ac turpis egestas. Mauris mattis consectetur quam, vel fringilla lorem ullamcorper eu \cite{blackholes}. Sed et arcu ante. Nulla pretium tellus id elit egestas, in pulvinar eros ornare. Suspendisse facilisis urna feugiat diam fringilla, eu viverra est viverra. Aliquam diam metus, mattis sit amet lorem id, blandit sagittis magna. Cras eu adipiscing lacus. Mauris ullamcorper tempor eros ac rhoncus. Ut scelerisque magna nibh, sed faucibus libero scelerisque eu. Aliquam ornare lacus eu metus sagittis, non condimentum mauris elementum.

Mauris pretium libero vitae auctor euismod. Donec porttitor aliquet ligula non commodo. Mauris hendrerit, dui quis ullamcorper adipiscing, ligula ligula tempus orci, in iaculis nisi metus a est. Duis lobortis neque mi, vel ullamcorper mauris gravida vel. In nunc libero, lacinia et dictum venenatis, fermentum ac nisi. Sed pellentesque molestie ipsum, ut posuere velit. Duis ac dui non lectus pharetra scelerisque. Vestibulum ante ipsum primis in faucibus orci luctus et ultrices posuere cubilia Curae; Cum sociis natoque penatibus et magnis dis parturient montes, nascetur ridiculus mus.

\section{Duis ac purus arcu}
Duis ac purus arcu. Nam posuere, neque eu elementum pretium, nulla ligula faucibus nisl, vitae rhoncus dui enim quis risus. Donec ullamcorper in dui eget rhoncus. Aliquam mollis ipsum sem. Donec tristique fermentum arcu, id lobortis neque scelerisque vitae. Duis cursus mi id lobortis ornare. Nunc et velit eu tellus cursus placerat. Praesent porttitor molestie ligula eu aliquet. Aliquam mattis libero non turpis lobortis laoreet. Integer sit amet rutrum nisi.

Pellentesque porttitor lorem eleifend nisl euismod, ut commodo erat varius. Morbi pellentesque iaculis nisl et ullamcorper. Suspendisse nec enim tincidunt, imperdiet nisl in, sagittis metus. Etiam in nulla aliquet, lacinia orci ac, condimentum nisl. Fusce luctus malesuada elit. In hac habitasse platea dictumst. Mauris mi eros, tempus sed blandit a, eleifend eget nisl. Quisque nisi nisl, blandit ut tincidunt sed, iaculis id ligula.

Aliquam at est at justo malesuada convallis. Mauris lectus ante, luctus vitae semper et, aliquet ac mauris. Nulla sit amet lectus ac massa rhoncus rutrum sed id ante. Etiam fringilla, tortor id accumsan euismod, dolor nisi vestibulum sapien, id dictum enim metus a turpis. Phasellus ligula nibh, viverra vitae lacinia ac, rhoncus vel felis. Nulla cursus luctus consectetur. Nam ut velit nec erat euismod gravida vel eget odio. Nulla fringilla sem ac justo aliquam, at lobortis diam condimentum. Integer venenatis tempus bibendum. Morbi vitae turpis velit. Aliquam eu posuere dui, et pellentesque sem. Cras dignissim sapien eros, ac egestas lectus convallis vel. Nulla condimentum ultricies velit. In consectetur leo eu fringilla egestas. 

Etiam gravida eget est pretium gravida. Curabitur id semper nunc. Fusce vel arcu id lorem hendrerit faucibus at nec nunc. Nam eget ligula orci. Quisque malesuada dui at tellus tincidunt, nec fermentum leo adipiscing. Lorem ipsum dolor sit amet, consectetur adipiscing elit. Praesent purus libero, lobortis ut neque at, sagittis dictum urna.

Donec vel lacinia augue. Sed gravida suscipit viverra. Nulla nibh purus, vehicula a elit vel, feugiat suscipit nibh. Mauris eu dapibus erat. Cras faucibus enim quam. Nulla elit dui, sollicitudin sed risus nec, tincidunt consequat sapien. Cras et nunc faucibus, pellentesque leo eget, fermentum augue. Sed euismod laoreet velit et posuere. Donec pellentesque rhoncus velit sit amet molestie. Sed sagittis condimentum dui, quis rhoncus libero dictum ut. Nulla vel bibendum mauris. Nulla arcu nulla, aliquam eu iaculis eu, mattis eu mauris. Maecenas elementum nec lacus vel feugiat. Ut nec lobortis erat, eu malesuada velit. Phasellus vel sem tincidunt sem pretium lacinia quis laoreet magna.

Morbi mi justo, ultricies adipiscing tortor ac, auctor mollis sapien. Maecenas venenatis vel orci sit amet consequat. Sed enim est, fringilla vel posuere vel, venenatis quis nibh. In euismod eu arcu quis pellentesque. Lorem ipsum dolor sit amet, consectetur adipiscing elit. Mauris ante sem, interdum at velit nec, lacinia venenatis mi. Sed quam metus, bibendum at mollis eu, vulputate in arcu. Morbi pulvinar metus risus, eu pharetra velit aliquet et. Praesent lacinia nisi metus, a varius dolor rutrum sed. Ut eu ultricies lacus, vulputate cursus odio. Sed feugiat felis et porta viverra. Ut odio massa, consequat vel feugiat venenatis, euismod id erat. Sed at nisi erat. Morbi eu quam tincidunt, semper nisi adipiscing, luctus ligula. Integer porta ipsum ut mi egestas luctus.

\begin{figure}
\begin{center}
\includegraphics[scale=1]{logoucvgifpeq.png}
\caption{Sello oficial de la UCV \label{selloUCV}}
\end{center}
\end{figure}

\section{Sed nec volutpat metus}
Sed nec volutpat metus. Cras egestas eros sit amet nibh egestas, non pellentesque metus porttitor. Maecenas mattis, ligula a dignissim rhoncus, justo turpis dignissim dolor, et tempus urna nibh ac erat. Suspendisse eu orci eros. Nunc nulla tellus, rutrum ut tempor vitae, laoreet quis massa. Cras pellentesque blandit mattis. Donec tristique a urna non posuere. Praesent porttitor tortor eget velit faucibus venenatis. Vivamus semper tincidunt aliquet. Curabitur est magna, lacinia ac tellus non, tempor convallis ipsum. Donec ullamcorper tellus eu tellus dictum sodales ut et turpis.

\begin{equation}
	\lim_{x \to \infty} \exp(-x) = 0
\end{equation}

Integer ut dapibus justo, id dictum erat. Nulla facilisi. Aliquam rhoncus mi feugiat volutpat rhoncus. Sed elementum tristique lectus ut faucibus. Morbi sit amet urna eget massa varius vulputate. Ut nisi ante, accumsan vitae justo in, convallis congue ligula. Duis varius mollis enim non ullamcorper. Class aptent taciti sociosqu ad litora torquent per conubia nostra, per inceptos himenaeos. Morbi felis sapien, bibendum at consequat nec, luctus quis dui. Nam porttitor libero dui, a tincidunt odio ullamcorper ac. Vivamus placerat, magna a eleifend molestie, est est eleifend lacus, ut congue magna est ut turpis. Sed porta auctor lacus, non placerat lorem imperdiet nec. Quisque et placerat ante. Donec ornare nunc felis, sed ultrices ante ultricies a. 

Lorem ipsum dolor sit amet, consectetur adipiscing elit. Quisque at tempor justo. Nullam dolor sapien, rutrum a ultrices eget, malesuada in est. Etiam et ipsum imperdiet, sollicitudin erat at, vehicula dui. Donec faucibus adipiscing magna vel ultrices. Morbi at urna et velit hendrerit consectetur vel eget erat. Morbi auctor imperdiet odio, in consequat neque varius ut. Pellentesque habitant morbi tristique senectus et netus et malesuada fames ac turpis egestas. Mauris mattis consectetur quam, vel fringilla lorem ullamcorper eu. Sed et arcu ante. Nulla pretium tellus id elit egestas, in pulvinar eros ornare. Suspendisse facilisis urna feugiat diam fringilla, eu viverra est viverra. Aliquam diam metus, mattis sit amet lorem id, blandit sagittis magna. Cras eu adipiscing lacus. Mauris ullamcorper tempor eros ac rhoncus. Ut scelerisque magna nibh, sed faucibus libero scelerisque eu. Aliquam ornare lacus eu metus sagittis, non condimentum mauris elementum.

Mauris pretium libero vitae auctor euismod. Donec porttitor aliquet ligula non commodo. Mauris hendrerit, dui quis ullamcorper adipiscing, ligula ligula tempus orci, in iaculis nisi metus a est. Duis lobortis neque mi, vel ullamcorper mauris gravida vel. In nunc libero, lacinia et dictum venenatis, fermentum ac nisi. Sed pellentesque molestie ipsum, ut posuere velit. Duis ac dui non lectus pharetra scelerisque. Vestibulum ante ipsum primis in faucibus orci luctus et ultrices posuere cubilia Curae; Cum sociis natoque penatibus et magnis dis parturient montes, nascetur ridiculus mus.

Duis ac purus arcu. Nam posuere, neque eu elementum pretium, nulla ligula faucibus nisl, vitae rhoncus dui enim quis risus. Donec ullamcorper in dui eget rhoncus. Aliquam mollis ipsum sem. Donec tristique fermentum arcu, id lobortis neque scelerisque vitae. Duis cursus mi id lobortis ornare. Nunc et velit eu tellus cursus placerat. Praesent porttitor molestie ligula eu aliquet. Aliquam mattis libero non turpis lobortis laoreet. Integer sit amet rutrum nisi.

\begin{itemize}
\item Los TEG se imprimen a doble cara 
\item Los márgenes superior e inferior son de 2,5 cm entre el papel y el texto 
\item En las páginas pares (izquierdas) el margen derecho es de 3 cm, el izquierdo es de 2,5 cm
\item La separaci\'on entre l\'ineas es de 18pt  
\item En las páginas impares (derechas) las dimensiones de los márgenes se invierten.
\item Los headers y footers son opcionales. En caso de usarse, las páginas izquierdas muestran el nombre del capítulo. Las derechas muestran el nombre de la sección.
\item Las páginas se numeran en la parte superior-exterior del libro (lado izquierdo en páginas pares y lado derecho en páginas impares)
\item La primera página de cada capítulo se imprime en una hoja impar. No se numera.
\item Todas las numeraciones de capítulos, secciones, ecuaciones, cuadros y figuras son arábicas. 
\item Las páginas iniciales (dedicatoria, resumen, etc) y los índices no se numeran, pero se cuentan en la numeración de las posteriores.
\item Las referencias se citan con números. La bibliografía se ordena siguiendo la secuencia de aparición de las citas en el texto.
\end{itemize}

%
\chapter{Pellentesque porttitor}

Pellentesque porttitor lorem eleifend nisl euismod \cite{vecchi1983}, ut commodo erat varius. Morbi pellentesque iaculis nisl et ullamcorper. Suspendisse nec enim tincidunt, imperdiet nisl in, sagittis metus. Etiam in nulla aliquet, lacinia orci ac, condimentum nisl. Fusce luctus malesuada elit. In hac habitasse platea dictumst. Mauris mi eros, tempus sed blandit a, eleifend eget nisl. Quisque nisi nisl, blandit ut tincidunt sed, iaculis id ligula.

Aliquam at est at justo malesuada convallis. Mauris lectus ante, luctus vitae semper et, aliquet ac mauris. Nulla sit amet lectus ac massa rhoncus rutrum sed id ante. Etiam fringilla, tortor id accumsan euismod, dolor nisi vestibulum sapien, id dictum enim metus a turpis. Phasellus ligula nibh, viverra vitae lacinia ac, rhoncus vel felis. Nulla cursus luctus consectetur. Nam ut velit nec erat euismod gravida vel eget odio. Nulla fringilla sem ac justo aliquam, at lobortis diam condimentum. Integer venenatis tempus bibendum. Morbi vitae turpis velit. Aliquam eu posuere dui, et pellentesque sem. Cras dignissim sapien eros, ac egestas lectus convallis vel. Nulla condimentum ultricies velit. In consectetur leo eu fringilla egestas. 

Etiam gravida eget est pretium gravida. Curabitur id semper nunc. Fusce vel arcu id lorem hendrerit faucibus at nec nunc. Nam eget ligula orci. Quisque malesuada dui at tellus tincidunt, nec fermentum leo adipiscing. Lorem ipsum dolor sit amet, consectetur adipiscing elit. Praesent purus libero, lobortis ut neque at, sagittis dictum urna.

\begin{equation}
\sqrt[n]{1+x+x^2+x^3+\ldots}
\end{equation}

Donec vel lacinia augue. Sed gravida suscipit viverra. Nulla nibh purus, vehicula a elit vel, feugiat suscipit nibh. Mauris eu dapibus erat. Cras faucibus enim quam. Nulla elit dui, sollicitudin sed risus nec, tincidunt consequat sapien. Cras et nunc faucibus, pellentesque leo eget, fermentum augue. Sed euismod laoreet velit et posuere. Donec pellentesque rhoncus velit sit amet molestie. Sed sagittis condimentum dui, quis rhoncus libero dictum ut. Nulla vel bibendum mauris. Nulla arcu nulla, aliquam eu iaculis eu, mattis eu mauris. Maecenas elementum nec lacus vel feugiat. Ut nec lobortis erat, eu malesuada velit. Phasellus vel sem tincidunt sem pretium lacinia quis laoreet magna.

Morbi mi justo, ultricies adipiscing tortor ac, auctor mollis sapien. Maecenas venenatis vel orci sit amet consequat. Sed enim est, fringilla vel posuere vel, venenatis quis nibh. In euismod eu arcu quis pellentesque. Lorem ipsum dolor sit amet, consectetur adipiscing elit. Mauris ante sem, interdum at velit nec, lacinia venenatis mi. Sed quam metus, bibendum at mollis eu, vulputate in arcu. Morbi pulvinar metus risus, eu pharetra velit aliquet et. Praesent lacinia nisi metus, a varius dolor rutrum sed. Ut eu ultricies lacus, vulputate cursus odio. Sed feugiat felis et porta viverra. Ut odio massa, consequat vel feugiat venenatis, euismod id erat. Sed at nisi erat. Morbi eu quam tincidunt, semper nisi adipiscing, luctus ligula. Integer porta ipsum ut mi egestas luctus.

\section{Cras egestas eros}
Sed nec volutpat metus. Cras egestas eros sit amet nibh egestas, non pellentesque metus porttitor. Maecenas mattis, ligula a dignissim rhoncus, justo turpis dignissim dolor, et tempus urna nibh ac erat. Suspendisse eu orci eros. Nunc nulla tellus, rutrum ut tempor vitae, laoreet quis massa. Cras pellentesque blandit mattis. Donec tristique a urna non posuere. Praesent porttitor tortor eget velit faucibus venenatis. Vivamus semper tincidunt aliquet. Curabitur est magna, lacinia ac tellus non, tempor convallis ipsum. Donec ullamcorper tellus eu tellus dictum sodales ut et turpis.


\begin{table}[h]
\caption{Ejemplo de tabla}
\begin{center}
    \begin{tabular}{ | l | l | l | p{5cm} |}
    \hline
    Day & Min Temp & Max Temp & Summary \\ \hline
    Monday & 11C & 22C & A clear day with lots of sunshine.  
    However, the strong breeze will bring down the temperatures. \\ \hline
    Tuesday & 9C & 19C & Cloudy with rain, across many northern regions. Clear spells
    across most of Scotland and Northern Ireland,
    but rain reaching the far northwest. \\ \hline
    Wednesday & 10C & 21C & Rain will still linger for the morning.
    Conditions will improve by early afternoon and continue
    throughout the evening. \\
    \hline
    \end{tabular}
\end{center}
\label{etiquetaTabla}
\end{table}

Integer ut dapibus justo, id dictum erat. Nulla facilisi. Aliquam rhoncus mi feugiat volutpat rhoncus. Sed elementum tristique lectus ut faucibus. Morbi sit amet urna eget massa varius vulputate. Ut nisi ante, accumsan vitae justo in, convallis congue ligula. Duis varius mollis enim non ullamcorper. Class aptent taciti sociosqu ad litora torquent per conubia nostra, per inceptos himenaeos. Morbi felis sapien, bibendum at consequat nec, luctus quis dui. Nam porttitor libero dui, a tincidunt odio ullamcorper ac. Vivamus placerat, magna a eleifend molestie, est est eleifend lacus, ut congue magna est ut turpis. Sed porta auctor lacus, non placerat lorem imperdiet nec. Quisque et placerat ante. Donec ornare nunc felis, sed ultrices ante ultricies a. 

\begin{itemize}
\item Los TEG se imprimen a doble cara 
\item Los márgenes superior e inferior son de 2,5 cm entre el papel y el texto 
\item En las páginas pares (izquierdas) el margen derecho es de 3 cm, el izquierdo es de 2,5 cm
\item La separaci\'on entre l\'ineas es de 16pt 
\item En las páginas impares (derechas) las dimensiones de los márgenes se invierten.
\item Los headers y footers son opcionales. En caso de usarse, las páginas izquierdas muestran el nombre del capítulo. Las derechas muestran el nombre de la sección.
\item Las páginas se numeran en la parte superior-exterior del libro (lado izquierdo en páginas pares y lado derecho en páginas impares)
\item La primera página de cada capítulo se imprime en una hoja impar. No se numera.
\item Todas las numeraciones de capítulos, secciones, ecuaciones, cuadros y figuras son arábicas. 
\item Las páginas iniciales (dedicatoria, resumen, etc) y los índices no se numeran, pero se cuentan en la numeración de las posteriores.
\item Las referencias se citan con números. La bibliografía se ordena siguiendo la secuencia de aparición de las citas en el texto.
\end{itemize}

%
%%%%%%%%%%%%%%%%%%%%
%% Apéndices
%%%%%%%%%%%%%%%%%%%
%
\appendix
\chapter{Integer ut}
Integer ut dapibus justo, id dictum erat. Nulla facilisi. Aliquam rhoncus mi feugiat volutpat rhoncus. Sed elementum tristique lectus ut faucibus. Morbi sit amet urna eget massa varius vulputate. Ut nisi ante, accumsan vitae justo in, convallis congue ligula. Duis varius mollis enim non ullamcorper. Class aptent taciti sociosqu ad litora torquent per conubia nostra, per inceptos himenaeos. Morbi felis sapien, bibendum at consequat nec, luctus quis dui. Nam porttitor libero dui, a tincidunt odio ullamcorper ac. Vivamus placerat, magna a eleifend molestie, est est eleifend lacus, ut congue magna est ut turpis. Sed porta auctor lacus, non placerat lorem imperdiet nec. Quisque et placerat ante. Donec ornare nunc felis, sed ultrices ante ultricies a. 

Lorem ipsum dolor sit amet, consectetur adipiscing elit. Quisque at tempor justo. Nullam dolor sapien, rutrum a ultrices eget, malesuada in est. Etiam et ipsum imperdiet, sollicitudin erat at, vehicula dui. Donec faucibus adipiscing magna vel ultrices. Morbi at urna et velit hendrerit consectetur vel eget erat. Morbi auctor imperdiet odio, in consequat neque varius ut. Pellentesque habitant morbi tristique senectus et netus et malesuada fames ac turpis egestas. Mauris mattis consectetur quam, vel fringilla lorem ullamcorper eu. Sed et arcu ante. Nulla pretium tellus id elit egestas, in pulvinar eros ornare. Suspendisse facilisis urna feugiat diam fringilla, eu viverra est viverra. Aliquam diam metus, mattis sit amet lorem id, blandit sagittis magna. Cras eu adipiscing lacus. Mauris ullamcorper tempor eros ac rhoncus. Ut scelerisque magna nibh, sed faucibus libero scelerisque eu. Aliquam ornare lacus eu metus sagittis, non condimentum mauris elementum.
 %cada capítulo incluido es un nuevo apendice
\chapter{Resumen de formato y medidas}
\begin{itemize}
\item Los TEG se imprimen a doble cara 
\item Los márgenes superior e inferior son de 2,5 cm entre el papel y el texto 
\item En las páginas pares (izquierdas) el margen derecho es de 3 cm, el izquierdo es de 2,5 cm
\item La separaci\'on entre l\'ineas es de 16pt 
\item En las páginas impares (derechas) las dimensiones de los márgenes se invierten.
\item Los headers y footers son opcionales. En caso de usarse, las páginas izquierdas muestran el nombre del capítulo. Las derechas muestran el nombre de la sección.
\item Las páginas se numeran en la parte superior-exterior del libro (lado izquierdo en páginas pares y lado derecho en páginas impares)
\item La primera página de cada capítulo se imprime en una hoja impar. No se numera.
\item Todas las numeraciones de capítulos, secciones, ecuaciones, cuadros y figuras son arábicas. 
\item Las páginas iniciales (dedicatoria, resumen, etc) y los índices no se numeran, pero se cuentan en la numeración de las posteriores.
\item Las referencias se citan con números. La bibliografía se ordena siguiendo la secuencia de aparición de las citas en el texto.
\end{itemize}
%
\cleardoublepage
\thispagestyle{empty}
\AddToShipoutPicture*{\BackgroundPic}
\begin{tabular}{cc} 
	\rule{0.67cm}{0ex} \rule{0cm}{3ex} & \rule{16.25cm}{0ex} \\ 
	\rule{5pt}{20ex} 3 cm  &
	\parbox[c]{8cm}{
	\sc \centering
	universidad central de venezuela\\
	facultad de ciencias\\
	escuela de física\\
	\rule{5pt}{10ex}} 1,5 cm\\ 
	\rule{5pt}{24ex} 3,6 cm & 
	\includegraphics[scale=1]{logoucvgifpeq.png}\\ 	\rule{5pt}{26ex} 3,9 cm &
		\raisebox{10ex}{\parbox[c]{15cm}{ \centering \tesistitulo }} \\ 
		%\raisebox{10ex}{\parbox[c]{15cm}{\sc \centering {\tesistitulo}}} \\ 
	\rule{5pt}{19ex} 2,9 cm &
	\parbox[c]{8cm}{			
		\centering
  		Trabajo Especial de Grado presentado por \\ 
 		\tesisautor \\
   		ante la Facultad de Ciencias de la \\
   		ilustre Universidad Central de Venezuela \\
   		como requisito parcial para optar al t\'{\i}tulo de: 
		\ifthenelse{\equal{\tesisautorgen}{m}}
   		{{\bfseries {Licenciado en F\'isica}}}
   		{\ifthenelse{\equal{\tesisautorgen}{f}}
   		{{\bfseries {Licenciada en F\'isica}}}
   		{{\bf Error! designación de genero de autor no admitida. Use letras minúculas m o f}}}\\   		
   		\begin{tabular}{rl}
   		\rule{0pt}{3ex}	Con la tutoría de:&\titulotutorA~ \nombretutorA\\
   		 						&\ifthenelse{\equal{\Dtutores}{si}}{ {\titulotutorB~ \nombretutorB} }{} 
   		\end{tabular}
  	}\\ 
	\rule{5pt}{8ex} 1,2 cm&
	\tesisanho \\ 
	\rule{5pt}{3ex} 0,5 cm &
	Caracas-Venezuela\\ 
\end{tabular}

%
%%%%%%%%%%%%%%%%%%%
%% Aquí viene la bibliografía
%% Se etiqueta con números y la bibliografía se lista por orden de aparición en el texto
%%%%%%%%%%%%%%%%%%%
\bibliographystyle{unsrt}
\bibliography{bibliografia}
\end{document}
